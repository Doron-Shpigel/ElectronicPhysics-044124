\documentclass{article}
\usepackage{settings}
%to write english paragraph \selectlanguage{language}
%to write english inline \textenglish{...}

\title{spring2018A}
\author{דורון שפיגל}
\date{02.04.24}

\begin{document}

\maketitle
\begin{Question}
בחומר מסוים נתונה הדיספרסיה של פס ההולכה:
\begin{equation*}
    \epsilon_{c}(k)= \epsilon_{0}+\frac{\hbar^2}{2}\left[ \left( \frac{k_{x}^4}{\alpha}-k_{x}^2 \right)\frac{1}{m_{1}}+\left( \frac{k_{y}^4}{\alpha}-k_{y}^2 \right)\frac{1}{m_{1}} + \frac{k_{z}^2}{m_{e}} \right]
\end{equation*}
נתון: $m_{1}=3m_{e}$, $\frac{\hbar^2\alpha}{2m_{1}}=1eV$ ו- $\epsilon_{0}=1eV$ והוא נמדד מהמקסימום של פס הערכיות הנמצא במרכז אזור ברילואין הראשון באנרגיה $\epsilon_{V}=0$.
\end{Question}
\begin{Answer}
תחילה אציב את הנתון $m_{1}=3m_{e}$ בנוסחה ואקבל:
\begin{align*}
        \epsilon_{c}(k)= \epsilon_{0}+\frac{\hbar^2}{2}\left[ \left( \frac{k_{x}^4}{\alpha}-k_{x}^2 \right)\frac{1}{3m_{e}}+\left( \frac{k_{y}^4}{\alpha}-k_{y}^2 \right)\frac{1}{3m_{e}} + \frac{k_{z}^2}{m_{e}} \right]\\
        \epsilon_{c}(k)= \epsilon_{0}+\frac{\hbar^2}{2m_{e}}\left[ \left( \frac{k_{x}^4}{\alpha}-k_{x}^2 \right)\frac{1}{3}+\left( \frac{k_{y}^4}{\alpha}-k_{y}^2 \right)\frac{1}{3} + k_{z}^2\right]
\end{align*}
נחשב את הגרדיאנט של הפונקציה ונמצא את הנקודות בהן היא מתאפסת:
\begin{align*}
    \nabla \epsilon_{c}(k)= \frac{\hbar^2}{2m_{e}}\left[ \left( \frac{4k_{x}^3}{\alpha}-2k_{x} \right)\frac{1}{3}+\left( \frac{4k_{y}^3}{\alpha}-2k_{y} \right)\frac{1}{3} + 2k_{z}\right]=0
\end{align*}
פתרון טריוויאלי הוא $k_{x}=k_{y}=k_{z}=0$. נבדוק את הנקודה הזו:
\begin{align*}
    \epsilon_{c}(0)= \epsilon_{0}+\frac{\hbar^2}{2m_{e}}\left[ 0+0+0\right]=\epsilon_{0}
\end{align*}
נחפש נקודות התאפסות על ידי השוואת מקדמים:
\begin{align*}
    \frac{4k_{x}^3}{\alpha}-2k_{x} = 0 \Rightarrow  k_{x}^2=\frac{\alpha}{2} &\Rightarrow k_{x}=\pm\sqrt{\frac{\alpha}{2}}\\
    \frac{4k_{y}^3}{\alpha}-2k_{y} = 0 \Rightarrow  k_{y}^2=\frac{\alpha}{2} &\Rightarrow k_{y}=\pm\sqrt{\frac{\alpha}{2}}\\
    2k_{z} = 0 &\Rightarrow k_{z}=0
\end{align*}
ולכן נקודות ההתאפסות של הגרדיאנט הן: ${\left\{ \vec{k}=\left( x,y,0 \right)\vert x,y = 0,\pm\sqrt{\frac{\alpha}{2}} \right\}}$.\\ עבור וקטורים מהצורות הבאות יש שיוויו מסימטריות משוואת הנפיצה בציר ה ${x,y}$. כך ש- 
\begin{align*}
    {\epsilon(\vec{k})=\epsilon((0,y,0))=\epsilon((x,0,0))}&= \epsilon_{0}+\frac{\hbar^2}{2m_{e}}\left[ \left( \frac{x^4}{\alpha}-x^2 \right)\frac{1}{3}\right]\\
    &=\epsilon_{0}+\frac{\hbar^2}{6m_{e}}\left[ \frac{\alpha}{2}^2\cdot\frac{1}{\alpha}-\frac{\alpha}{2} \right]\\
    &= - \frac{\hbar^{2} \alpha}{24 m_{e}} + \epsilon_{0} \underbrace{<}_{\frac{\hbar^2\alpha}{2m_1}=1>0}\epsilon_{0}\\
    \epsilon(\vec{k})=\epsilon((x,x,0))&=\epsilon_{0}+\frac{\hbar^2}{6m_{e}}\left[ \frac{\alpha}{2}^2\cdot\frac{1}{\alpha}-\frac{\alpha}{2} +\frac{\alpha}{2}^2\cdot\frac{1}{\alpha}-\frac{\alpha}{2} \right]\\
    &= - \frac{\hbar^{2} \alpha}{12 m_{e}} + \epsilon_{0} \underbrace{<}_{\frac{\hbar^2\alpha}{2m_1}=1>0}\epsilon_{0}
\end{align*}
כלומר:
\begin{equation*}
    \underbrace{\epsilon((x,x,0))}_{=- \frac{\hbar^{2} \alpha}{12 m_{e}} + \epsilon_{0}}<\underbrace{\epsilon((x,0,0))}_{=- \frac{\hbar^{2} \alpha}{24 m_{e}} + \epsilon_{0}}<\underbrace{\epsilon((0,0,0))}_{=\epsilon_{0}}
\end{equation*}
ולכן נקודות המינימום של פס ההולכה הן: $\vec{k}=\left( \pm\sqrt{\frac{\alpha}{2}},\pm\sqrt{\frac{\alpha}{2}},0 \right)$.\\
נתון כי $\epsilon_{0}=1eV,\quad \epsilon_{V}=0eV$.
\begin{align*}
    E_{gap}&=E_{Cmin}-E_{Vmax}=E_{Cmin}-0=E_{Cmin}\\
    &=  \frac{-\hbar^{2} \alpha}{12 m_{e}} + \epsilon_{0}=\frac{1}{2}\cdot \frac{-\hbar^{2} \alpha}{2 m_{1}}+\epsilon_{0}=\frac{-1}{2}eV+1eV=0.5eV
\end{align*}
נתון כי המקסימום של פס הערכיות נמצא במרכז אזור ברילואין הראשון באנרגיה $\epsilon_{V}=0$.כלומר:
$E_{Vmax}(\vec{k})=0\leftrightarrow \vec{k}=0$. לעומת זאת, ראינו ש $E_{Cmin}(\vec{k})\leftrightarrow \vec{k}\neq0$, לכן פער האנרגיה אינו ישר.\\ החומר אינו יכול לשמש עבור רכיבים פולטי אור מהסיבה שפוטונים מהווים מעברים כמעט אנכיים, ומכיוון שרק קצוות הפסים מאוכלסים, לא תוכל להתקיים פליטת פוטונים.\\
כעת נמצא את צפיפות המצבים \underline{בתחתית} פס ההולכה: כיוון שמדובר בתחתית הפס, נרצה לבצע קירוב פרבולי לפס האנרגיה בנקודות המינימום $\vec{k}=\left( \pm\sqrt{\frac{\alpha}{2}},\pm\sqrt{\frac{\alpha}{2}},0 \right)$. לפי משוואת הנפיצה הנתונה, ציר $\hat{z}$ פרבולי לחלוטין: $\left( \frac{\hbar^2}{2m_e}k_{z}^{2} \right)\hat{z}$, ואילו על צירי $\hat{x},\hat{y}$ נצטרך לבצע קירוב טיילור (מסדר שני).
\label{טור טיילור מסדר 2}
\begin{align*}
    \eval{f(x)}_{x=a}&\approx f(a)+f'(a)(x-a)+\frac{f''(a)}{2}(x-a)^2
    \tag{\textenglish{2nd order taylor for single variable}}\\
    \eval{f(x,y)}_{x=a,y=b}&\approx f(a,b)\\
    &+\eval{\od{f(x,y)}{x}}_{x=a,y=b}(x-a)+\eval{\od{f(x,y)}{y}}_{x=a,y=b}(y-b)\\
    &+\eval{\od[2]{f(x,y)}{x}}_{x=a,y=b}\frac{(x-a)^2}{2}+\eval{\od[2]{f(x,y)}{y}}_{x=a,y=b}\frac{(y-b)^2}{2}\\
    &+\eval{\od{}{y}\od{f(x,y)}{x}}_{x=a,y=b}(x-a)(y-b)
    \tag{\textenglish{2nd order taylor for two variables}}
    %(a,b)+fx(a,b)(x−a)+fy(a,b)(y−b)+fxx(a,b)2(x−a)2+fxy(a,b)(x−a)(y−b)+fyy(a,b)2(y−b)2
\end{align*}
אחשב כל גורם של הסכום עבור $f(x,y)=\epsilon_{c}(k_{min})$:
\begin{align*}
    \epsilon_{c}(k_{min})&= \epsilon_{0}+\frac{\hbar^2}{2m_{e}}\left[ \left( \frac{k_{x}^4}{\alpha}-k_{x}^2 \right)\frac{1}{3}+\left( \frac{k_{y}^4}{\alpha}-k_{y}^2 \right)\frac{1}{3} + k_{z}^2\right]\\
    &= \epsilon_{0}-\frac{\hbar^2\alpha}{12m_e}=0.5eV=E_{gap}\\
    \eval{\od{\epsilon_{c}(k_{min})}{k_{x}}}_{\begin{matrix}
        a=k_{xmin}\\
        b=k_{ymin}
    \end{matrix}}\left( x-k_{xmin} \right)&= 
    \frac{\hbar^2}{2m_{e}}\left[ \left( \frac{4k_{xmin}^3}{\alpha}-2k_{xmin} \right)\frac{1}{3}\right]\left( x-k_{xmin} \right)=0\\
    \eval{\od{\epsilon_{c}(k_{min})}{k_{y}}}_{\begin{matrix}
        a=k_{xmin}\\
        b=k_{ymin}
    \end{matrix}}\left( y-k_{ymin} \right)&=0\\
    \eval{\od[2]{\epsilon_{c}(k_{min})}{k_{x}}}_{\begin{matrix}
        a=k_{xmin}\\
        b=k_{ymin}
    \end{matrix}}\frac{(x-k_{xmin})^2}{2}&=\frac{\hbar^2}{2m_{e}}\left[ \left( \frac{12k_{xmin}^2}{\alpha}-2\right)\frac{1}{3}\right]\frac{(x-k_{xmin})^2}{2}\\
    &=\frac{\h^2}{2m}\left[ \left( \frac{12k^2}{\alpha}-2\right)\frac{1}{3}\right]\frac{(x-k)^2}{2}
\end{align*}
\end{Answer}














\end{document}