\documentclass{article}
\usepackage{settings}
%to write english paragraph \selectlanguage{language}
%to write english inline \textenglish{...}

\title{spring2023A}
\author{דורון שפיגל}
\date{13.04.24}

\begin{document}
\maketitle



\begin{Question}
נתון חומר תלת מימדי עם יחס נפיצה הדומה לגרפן: $E=\hbar V_{f}\sqrt{k_{x}^{2}+k_{y}^{2}+k_{z}^{2}}$. הספין של האלקטרון בערך מוחלט הוא $\frac{3}{2}$.\\
מספר המצבים ליחידת אנרגיה וליחידת נפח נפח הוא?
מספר המצבים עד מספר גל $k$ עבור מערכת באורך $L$ הוא:
\begin{align}\label{מספר מצבים}
    N\left( k \right)&=2\cdot\frac{k}{\frac{\pi}{L}}\tag{1d}\\
    N\left( k \right)&=2\cdot\frac{\pi k^{2}}{4\left( \frac{\pi^2}{L^{2}} \right)}\tag{2d}\\
    N\left( k \right)&=2\cdot\frac{\frac{1}{8}\cdot\frac{4\pi}{3}k^{3}}{\left( \frac{\pi}{L} \right)^{3}}\tag{3d}
\end{align}
נחלץ את $k$:
\begin{align*}
    E=\hbar V_{f}\sqrt{k_{x}^{2}+k_{y}^{2}+k_{z}^{2}}=\hbar V_{f}\abs{\vec{k}}\\
    \abs{\vec{k}}=\frac{E}{\hbar V_{f}}
\end{align*}
המערכת תלת מימדית לכן עקום האנרגיה הוא:
\begin{align*}
    \sum \left( \epsilon \right)=\frac{4\pi}{3}k^{3}=\frac{4\pi}{3}\frac{\epsilon^{3}}{\left( \hbar V_{f} \right)^{3}}\\
    V_{system}=L^{3}\\
    V_{singale-state}=\left( \frac{\pi}{L} \right)^{3}
\end{align*}
נוסחה לצפיפות מצבים:
\begin{equation}
    g\left( \epsilon \right)=\frac{d}{d\epsilon}\underbracket[0.1ex]{\left[ 
        \text{\footnotesize{\#degenerate}}\cdot\sum\left( \epsilon \right)\cdot \frac{1}{V_{single-state}\cdot \text{\footnotesize{\#sub-lattice}}}
        \cdot \underbracket[0.1ex]{2}_{\text{spin}}\cdot \frac{1}{2^{\left\{ d:0,2,3 \right\}}}\cdot
        \frac{1}{V_{system}}
    \right]}_{=n}
\end{equation}
\begin{align*}
    n&=\frac{4\pi}{3}\frac{\epsilon^{3}}{\left( \hbar V_{f} \right)^{3}}\cdot \frac{1}{\frac{\pi^{3}}{L^{3}}}\cdot\frac{3}{2}\cdot\frac{1}{8}\cdot\frac{1}{L^{3}}\\
    \frac{4\pi}{3}\cdot \frac{1}{\frac{\pi^{3}}{L^{3}}}\cdot\frac{1}{8}\cdot\frac{1}{L^{3}}\cdot2 = \frac{2}{3 \pi^{2}}
    &=\frac{\epsilon^{3}}{\left( \hbar V_{f} \right)^{3}}\frac{1}{4 \pi^{2}}\\
\end{align*}
\end{Question}











\end{document}